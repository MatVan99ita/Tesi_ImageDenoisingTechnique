\documentclass[12pt,a4paper,openright,twoside]{book}
\usepackage[utf8]{inputenc}
\usepackage{algorithm}
\usepackage{algpseudocode}
\usepackage{amsmath}
\usepackage{amsthm}
\usepackage{amssymb}
\usepackage{bm}
\usepackage{colortbl}
\usepackage{graphicx}
\usepackage{geometry}
\usepackage{hyperref}
\usepackage{ifthen}
\usepackage{latexsym}
\usepackage{listings}
\usepackage{makeidx}
\usepackage{tikz}
\usepackage{titlesec}
\usepackage{wrapfig}
\usepackage{xcolor}
\definecolor{mygreen}{rgb}{0,0.6,0}
\definecolor{mygray}{rgb}{0.5,0.5,0.5}
\definecolor{mymauve}{rgb}{0.58,0,0.82}
\lstset{
basicstyle=\ttfamily,
language=Python,
commentstyle=\color{mygreen},
keywordstyle=\color{blue},
numberstyle=\tiny\color{mygray},
stringstyle=\color{mymauve},
numbers=left,
showstringspaces=false
extendedchars=true
}
\newcommand{\qvec}[1]{	extbf{	extit{#1}}}
\newcommand{\mathbi}[1]{	extbf{\em #1}}
\newcommand{\TODOComment}[1]{}
\newcommand{\underlinebrace}[2]{%
\tikz[remember picture,baseline=(a.base)]{%
\node[inner sep=0pt] (a) {#1}; 
\draw[decorate,decoration={brace,mirror,amplitude=4pt}] (a.south west) -- (a.south east);
\node[below=4pt] at (a.south) {#2};
}%
}
\begin{document}
\title{Post-Quantum Cryptography}
\author{Matteo Vanni}
\date{\today}
 
\newgeometry{margin=0.8in}
\begin{titlepage}
\begin{center}
 
    \large
    \textbf{ALMA MATER STUDIORUM -- UNIVERSITÀ DI BOLOGNA\\CAMPUS DI CESENA}
\\
    \noindent\hrulefill
    \vspace{0.4cm}
 
    \Large
    Corso di Laurea triennale in Ingegneria e Scienze Informatiche\\
 
    \Huge
    \vspace{4.5cm}
    \Large
    \textbf{
        Image Denoising Technique with Neural Network
   }
 
    \large
    \vspace{1cm}
 
    \vspace{5.5cm}
    \begin{minipage}[t]{0.64\textwidth}
      \begin{flushleft}
        \textit{Relatore:}\\
        \textbf{Prof. Lazzaro Damiana}
        \\
        \vspace{0.4cm}
        \textit{Co/Contro Relatore}\\
        \textbf{Dott. Ezio Greggio}
      \end{flushleft}
    \end{minipage}
    \begin{minipage}[t]{0.34\textwidth}
      
      \begin{flushright}
        \textit{Candidato:}\\
        \textbf{Matteo Vanni}
\\
        \textit{Matricola: 0000935584}
      \end{flushright}

    \end{minipage}\\
 
    \vfill
    \noindent\hrulefill

    \textit{MMMCAINA 20256}
    \vspace{0.3cm}
    \Large
 
  \end{center}
\end{titlepage}
 
\titleformat{\chapter}{\normalfont\huge \bfseries}{\Huge \thechapter}{20pt}{\Huge}
\tableofcontents

\chapter{Articolanzio}
\section{Panoramica sul problema}
L'image denoising è il processo di rimozione di rumore da un'immagine.\\
Il rumore, che è causato da svariate fonti, quali foto fatte in condizioni di scarsa illuminazione o problemi che corrompono i file, causa perdita d'informazione sull'immagine.
\paragraph{Cos è il rumore?} 
Un aggiunta casuale di pixel che non appartengono all'immagine originale e ce ne sono di varie tipologie:\\
Impulse Noise(IN) dove i pixel sono completamente diversi da quelli attorno. Esistono due categorie di IN: Salt and Pepper Noise(SPN) e Random Valued Impulse Noise(RVIN).\\
Additive White Gaussian Noise(AWGN) cambia ogni pixel dall'originale di una piccola quantità.\\

\section{Utilizzo di modelli di Deep Learning}
\'E essenziale rimuovere il rumore e ristabilire l'immagine originale dove 
riottenere l'immagine originale è importante per prestazioni robuste o ricostruire le informazioni mancanti è molto utile, come immagini astronomiche di oggetti molto lontani.\\
Le reti neurali convoluzionali lavorano bene con le immagini e ne utilizzeremo N, menzionate in alcuni paper di ricerca e compareremo i risultati di ogni modello.\\

\section{Dataset utilizzati}
gli animali di arvard, le colonscopie \TODOComment{Approfondire aggiungendo link ai dataset e approfondire cose che non ci sono}
AAAAAAAAAAAAAAAAAAAAAAAAAAAAAAAAAAAAAAAAAAAAAAAAAAAAAAAAAAAAAAAA
\chapter{Decsrizione delle reti}
\section{Rumore}
E FAI RUMORE \textbf{QUIIIIIIIIIIIIIIIIIIIIIIIIIIIII}
Prima funzione di rumore -> Random noise con fattore 0.3 massimo
\TODOComment{questi 2 più avanti}
Rumore gaussiano 
Rumore a più layer


\section{Autoencoder}
Il primo approccio è stato quello di usare l'autencoder dell'articolo da cui sto copiando paro paro tutto.\\
Spiegazione del numero di layer usati e del tipo di mse loss fun, dam e learning rate di 1e-3

\TODOComment{Modificare il dataset per i cosi successivi}

\section{RIDNet}
stessa cosa dell'autencoder (vedere se aggiungere tutto il codice gigante del setup)


\chapter{Analisi ed ottimizzazione}
\section{Prestazioni}
PSNR è il metodo più comune per misurare la qualità delle immagini.\\
Il PSNR è definito come il rapporto tra il massimo valore possibile di un segnale e il valore del rumore che disturba la qualità della sua rappresentazione.\\ Normalmente misurato in una scala logaritmica in decibel(dB).\\
Data l'immagine originale(\textit{g}) e l'immagine rumorosa(\textit{f}), il PSNR è definito come: \[PSNR=20log_{10}(\frac{MAX_f}{\sqrt{MSE}})\] dove $MAX_f$ è il valore massimo del pixel dell'immagine e si calcola come \[MSE=\frac{1}{mn}\sum_{0}^{m-1}\sum_{0}^{n-1}||f(i,j)-g(i,j)||^2  \] mentre $MSE$(\textit{Mean Square Error}) è l'errore quadratico medio tra l'immagine originale e quella rumorosa.\\
\TODOComment{Aggiungere codice del pnrr}



\section{Quantizzazione dei modelli}
\section{Analisi dei modelli}
\subsection{Performance su dataset sconosciuti}



\chapter{Bibliografia}
\begin{itemize}
    \item \href{<sito>}{<nome>}
\end{itemize}
\end{document}




